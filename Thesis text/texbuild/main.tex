\documentclass[12pt]{report}
\usepackage[utf8]{inputenc}
\usepackage{graphicx}
\graphicspath{{images/}}
\usepackage[T1]{fontenc} % Output font encoding for international characters
\usepackage[a4paper,width=150mm,top=25mm,bottom=25mm,bindingoffset=6mm]{geometry}
\usepackage{mathpazo} % Palatino font

\usepackage{fancyhdr}
\pagestyle{fancy}

\fancyhead{}
\fancyhead[RO,LE]{Sportconstraint Learning using Tensor Data}
\fancyfoot{}
\fancyfoot[LE,RO]{\thepage}
\fancyfoot[LO,CE]{Chapter \thechapter}
\fancyfoot[CO,RE]{Tom De Bièvre}
\renewcommand{\headrulewidth}{0.4pt}
\renewcommand{\footrulewidth}{0.4pt}


\begin{document}
\input{titlepage.tex}


% pre table
\chapter*{Abstract}
\chapter*{Dedication}
\chapter*{Acknowledgements}

\tableofcontents

\chapter{Introduction}
Many real world problems could be easily solved using constraints. In this paper we take a particular interest in the world of sport scheduling. The sport scheduling problem consists of planning games in a sport tournament that satisfy a certain amount of constraints stated by different stakeholders within the tournament (organizers, participants, television rights, fans, government, …).
\\[5px]
A wide variety of scheduling models exists with regards to formats of sporting tournaments. A brief summary will be given, but the main focus of this paper will be on round robin tournaments due to their simplicity and resulting from that, the fact that they are very commonly used in different sports. A round robin tournament is a tournament where each player or team plays each one of the other players or teams a fixed number of times.
\\[5px]
Apart from planning when teams play each other, the location is also important. A core aspect of sport scheduling is the creation of schedules which are optimized for logistics and offer every stakeholder the same amount of (dis)advantages.
\\[5px]
Trying to obtain a realistic model that is both optimized as well as satisfying all the constraints of the stakeholders is very hard and labour-intensive [1]. An alternative to solving this problem could be the employment of constraint learning to induce a model based on given positive examples. A given model could be used “as is” to produce solutions in line with the problem, or serve as a base model to design the final model.
\\[5px]
There are many artificial intelligence approaches that could solve the components of this problem. The hard constraints stated by the stakeholders could easily be solved with basic constraint programming, while the optimizations could be satisfied using integer programming [2]. The sport scheduling problem involves numerical constraints. For instance, the number of consecutive home or away games for a certain team could be subject to a constraint.  Classical learning approaches like Conacq and Inductive Logic Programming focus on boolean variable, and it -s unclear how we could introduce numerical terms in here.
\\[5px]
The sport constraint problem is inherently multi-dimensional, so the constraint learning approach used to solve the sport scheduling problem is based on COUNT-OR [3]. This approach is used as a benchmark. The adaptation Count-SPORT is designed as a generalization of Count-OR, trying to acquire hard constraints like “team 2 plays not more than 2 consecutive games at home”, but also tries to optimize schedules based on soft constraints.

\include{chapters/0.Introduction.tex}

\chapter{Constraint satisfaction problems}
\section{Constraint Satisfaction Problems}

\subsection{Constraint satisfaction problem}
The constraint satisfaction problem(CSP) consists in finding values for variables such that a set of constraints are satisfied. This is used to express and solve many real world problems, like the scheduling of sport tournaments. 

A constraint network N = (X,D,C) consists of:
\begin{itemize}
    \item A finite set of variables X : {$X_1$,…,$X_n$}
    \item A set of domains D: {$D(X_1),…,D(X_n)$} where $D(X_i)$ is the (finite) set of values $X_i$ can take, in this work, every domain is assumed finite
    \item And a set of constraints C: {$c_1, .. c_n$} where each constraint c \in C is a pair c = $(\sigma, \rho)$
    \begin{itemize}
        \item $\sigma$ is the constraint \textbf{scope}, a set of variables
        \item $\rho$ the constraint \textbf{relation} a subset of the Cartesian product of the domains in the scope
    \end{itemize}
\end{itemize}

\subsection{ Terminology}
An \textbf{assignment} is a pair ($x_i$,a) which means $x_i \in X$ is assigned the value $a \in D_i$. A \textbf{compound assignment} is a set of assignments to distinct variables in X.

The \textbf{relation} of a constraint c =  ($\sigma_c, \rho_c$) specifies all the acceptable assignments to the variables in its scope. If the constraint scope $\sigma_c$ is ${xi_1 , x_i2 , ..., x_ik}$ and ${a_1, a_2, ..., a_k}$ \in $\rho_c$, the compound assignment assigning $a_i$ to $x_{ik}$ , $1 \leq i \leq k$, is an acceptable assignment; we say that the assignment \textbf{satisfies} the constraint c.

We consider a constraint \textbf{satisfiable} if there exists an assignment of values $v_i \in D(X_i)$ for each $X_i$ such that the constraint satisfies; and we consider the constraint \textbf{unsatisfiable} if that assignment does not exist. 

The \textbf{arity} of a constraint is the size of its scope. A unary constraint is defined on a single variable. A binary constraint on two variables. There are no requirements that different constraints have different scopes, so different constraints could have the same scope.

A \textbf{solution} to a constraint network is an assignment of values to each variable in X such that every $c \in C$ is satisfied. 


\chapter{Different Sport Tournaments}
Many different kinds of formats could be used for the organization of (sport) tournaments. In this section we try to list the more common forms. Important to note: the terminology within the existing literature is far from consistent. Several terms exist for the same concept, and some frequently used terms represent multiple concepts. Despite the fact that the authors of different sources referenced in this thesis use different terminology, the terminology in this section and the next will be used throughout this thesis.

\section{Ladder-Based Tournaments}
\subsection{Basic ladder tournament}
In a ladder tournament, participants are listed on the rungs of a ladder in current order of merit, with the best player on top of the list. Competition is challenge-based: players are allowed to challenge players above them on the ladder. Usually, a limit exists of how many rungs above them players can challenge. If the lowest-placed player wins the match, then the two players exchange places on the ladder. If the highest-placed player wins, he is allowed to challenge a player above him before accepting another challenge. Rules about challenges need to be stated beforehand.
\\[5px]
This format is used in sports like squash and badminton. The most famous ranking system for players is the Elo rating system, which is used in Chess.
\\[5px]
Since tournaments based on this format don’t have a formal schedule, we won’t elaborate on it any further.

\subsection{Pyramid tournament}
Pyramid tournaments are fairly similar to ladder tournaments insofar that they also maintain continuous, prolonged forms of competition, but this form allows for more challenges to be made and thus, more participation. They can also include a larger number of participants than the ladder tournaments.
\\[5px]
After the preliminary draw, players can challenge any other participant on the same horizontal row of the pyramid. If they win, they can challenge a participant on a higher row. If a participant loses from someone on a lower row, they switch places within the pyramid. Just like with ladder tournaments, challenge rules need to be stated beforehand.
\\[5px]
Since pyramid formats also lack a formal schedule, we won’t elaborate on them any further.

\section{Elimination-based tournaments}
\subsection{Single elimination tournament}
Single elimination tournaments or single knockout tournaments are the simplest forms of tournaments: the winner of each match advances in the tournament tree and the loser gets eliminated. This means that after only one loss, a participant is completely eliminated. No provision for off-day’s or bad luck for a participant are present. Single elimination tournaments prove their usefulness when faced with a large number of contestants and only a short period of time. They also find their use in final stages of big sports tournaments where games have high stakes (e.g.: finals of World Championship football).
\\[5px]
If all participants are considered equal, the seeding of the tournament tree is random. If participants have known abilities/scores, they are seeded in such a way that the strong participants avoid each other as long as possible (e.g.: in the knockout phase of the Champions league football, the winners of the group phase can’t play each other in the first knockout game).  

\subsection{Consolation tournament}
A consolation tournament or ‘losers-bracket’ usually goes hand in hand with a single elimination tournament. When participants lose in a round, they advance to the loser bracket where they compete each other in a single elimination principle for the consolation title. A ‘feed-in’ consolation tournament enables losers from the first round up to losers of the quarter finals to compete for the consolation title. The most famous example is judo, where in big tournaments 2 bronze medals are awarded (one for the winner of the consolation tournament, and one for the best of the two losing semi-finalists).

\subsection{Double Elimination tournament}
Double elimination tournaments consist of two games. As the name suggests, a participant must lose twice before getting eliminated. It is preferred over the single elimination format when there aren’t as many players involved. Due to the requirement for two losses, this format allows players to have an “off-day”. This tournament model isn’t commonly used, but finds its application in table football and e-sports.
\\[5px]
Since elimination tournaments don’t really have lots of edge cases that are also hard constraints, we won’t discuss this type of tournament any further.
\section{Round Robin tournaments}
\subsection{Straight round robin}
Round robin tournaments / league schedules consist of a format whereby every participant is playing every other participant a fixed number of times. In a single round robin (1RR/ SRR) tournament, every participant plays every other participant only once. 1RR is a very well-known format and commonly used for group stages in big sports tournaments. Examples are: the FIFA world cup, continental tournaments, …
\\[5px]
In a double round robin (2RR/DRR) tournament, every participant plays every other participant twice. Double round robin is the most common tournament schedule in (professional) sports leagues. Triple and quadruple round robin also exist.
\\[5px] 
Throughout this paper, we will use the notation xRR. The notion of x-round robin means that every participant plays every other participant x number of times.
\\[5px]
Round robin schedules are fixed: all participants know from the start when and where they will play a particular match. The cumulative result of all the games determines who the winner is of the tournament, so there is no concept of a “final” in this model.
\\[5px]
Round robin is a very effective scheduling model for one-day tournaments, where games can be played fast and a small number of participants is involved. When there is a larger number of participants and the games take longer, you want to use round robin for league play.
\\[5px]
The following numbers will illustrate that round robin isn’t ideal when faced with a large number of participants: in a tournament with 32 participants, it would take 496 games to complete a round robin, where double elimination would take 62 games and single elimination only 31.    

\subsection{Round robin Double split}
When a RR format is desired, but the number of participants is too large, splitting them into two divisions can offer a practical solution. Following the play within the divisions, only the top two of each division advances to the play-offs to determine the final top standings. The number of games gets halved this way. The drawback of this format is that accurate seedings of the divisions becomes very important. For example, if the top three seeds are placed in one division, and only the top two advances to the playoffs, then (if entries perform consistent with their seedings) the third seed can’t play in the play-offs.
\\[5px]
Round robin double split finds its uses in league play. An example is play-off II of the Belgium football competition, where you have two divisions of four teams, and the winner of each division advances to a play-off where they could go for a ticket to the Europa League.
\subsection{Round Robin Triple split}
RR triple split is very similar to the double split. However, because of the three divisions, single elimination playoff with three or six games becomes rather impossible. So, for playoffs in this situation, we use a new RR tournament. This will require more games in the playoffs, but makes it also a satisfactory alternative for RR double split when there are too many participants.
\subsection{Round Robin Quadruple split}
This type of tournament is intended to solve the same problems as RR double split, but instead of dividing into two divisions, we divide into four. This is useful when the number of entries exceeds 11. The major disadvantage of this format is that with 12-15 participants, the weaker seeds conceivably could only play 2 games.
\subsection{Note on Round Robin}
Note that you could describe several complex tournament schedules as consecutive RR-tournaments with a different number of participants. Take the example of the Belgium football competition with playoffs (2009). The regular competition consists of a 2RR tournament with 16 teams. The first six teams advance to play-off I, then play a 2RR tournament to decide who wins the championship. Teams 8-14 get seeded into a RR Double split format, where the winner of each division plays single elimination to decide who wins playoff 2. Team 15-16 play a double elimination tournament to decide who relegates to the lower division.
\\[5px]
With RR being the more common tournament form in professional leagues and the fact that consecutive RR-like tournaments could be used to model complex sport competitions, we take a particular interest in RR for this thesis.


\chapter{Terminology}
\section{Round Robin Specific}
\subsection{Round}
When scheduling a round robin tournament, the games are divided into different rounds in such a way that every team plays maximum one game each round. In the literature this is referred to as a round, a slot, a timeslot, … It is not mandatory that a round takes place at a fixed time. A round could take multiple days (e.g. a complete weekend).
\\[5px]
 If the amount of participants n is even, we need to schedule at least n-1 rounds to schedule a SRR tournament. When n is odd, we need at least n rounds. If the number of rounds equals this lower bound, we call our tournament \textbf{compact}. If we have more rounds, it’s called \textbf{relaxed}. For instance, the majority of football tournaments have compact schedules, while relaxed schedules dominate in basketball. In this thesis, we will assume compact tournaments.
 
\subsection{Cycle}
 We define a cycle as the collection of rounds necessary for each participant to play every other participant. In other words, there is only one cycle in a SRR tournament, two cycles in a DRR tournament, and x cycles in an xRR tournament. Assuming the given schedule is compact, a cycle will always be n-1 rounds, with n the number of participants. Literature that focuses on cycles in Round Robin tournaments is somewhat lacking, so few other terms exist for this concept.
 
\subsection{Home Away Pattern(HAP)}
In literature, participants often have a designated home base. Since the more common use of round robin is within the world of team sports (or sports leagues where home-advantages actually matters), we will for the purpose of this research assume that participants of a round robin always have a home base.
\\[5px]
When a participant plays a game at his home base it is called a \textbf{home game}, and when a participant plays at the home base of the opposing team, it is called an \textbf{away game}. When a participant doesn’t play in a round, he has a bye. We can therefore assume that when two participants play each other, one plays home and the other plays away. However, this isn’t always strictly the case. Consider for example a sports tournament that takes place on neutral ground (e.g. World Championship football). The sequence of home games, away games and byes for each team is a \textbf{home-away pattern (HAP)}.
\\[5px]
The modeling of a HAP consists of a sequence of \textbf{H}(ome)’s, \textbf{A}(way)’s and \textbf{B}(ye)’s with a length of n-1. This sequence corresponds with the home and away sequence of a participant (n the amount of teams).
\\[5px]
There are no byes in a compact tournament unless the number of participants is odd.  The combination of HAP’s for every participant is called a \textbf{HAP set}. A HAP set is \textbf{equilibrated} or \textbf{balanced} if the number of home games for each team doesn’t differ more than 1. Two patterns are \textbf{complementary} if the first pattern has a H when the second has an A and vice versa. An example of a balanced and complementary set of HAP’s is shown in table 1.

\begin{table}[h]
\centering
\begin{tabular}{|c|c|}
 \hline
 Team1 & HAHAHAHAHA \\ 
 \hline 
 Team2 & AHAHAHAHAH \\
 \hline
 Team3 & AHHAHHAAHA \\
 \hline
 Team4 & HAHAAAHAHH \\
 \hline
 Team5 & AHAHHHAHAA \\
 \hline
 Team6 & HAAHAAHHAH \\
 \hline
\end{tabular}
\caption{Table 1: Balanced \& complementary set of HAP’s for a compact 2RR tournament}
\end{table}

\subsection{Break}
Many tournaments aim for an alternating pattern of home and away games. When a team has two consecutive home / away games. It is called a \textbf{break}. If all teams have the same number of breaks, the HAP set is called \textbf{equitable}. If we look at table 1, only teams 1 and 2 have no breaks. This means the HAP set in table 1 is non-equitable. Some leagues want to minimize the number of breaks, but in some cases it could be preferable for participants to play consecutive games away [6].
\\[5px]
The latter is the case when opponents’ home bases are far removed from one another and where minimizing total travel distance is a point of optimization. Consecutive games away are called a road trip.  Consecutive games at home are a home stand.
\\[5px]
The assignment of rounds could be represented in a time table. Each row represents a team and each column represents a round.

\begin{table}[!h]
\centering
\begin{tabular}{|l|l|l|l|l|l|l|l|l|l|l|}
\hline
\textbf{Round}  & \textbf{1} & \textbf{2} & \textbf{3} & \textbf{4} & \textbf{5} & \textbf{6} & \textbf{7} & \textbf{8} & \textbf{9} & \textbf{10} \\ \hline
\textbf{Team 1} & 6          & 3          & 5          & 2          & 4          & 6          & 3          & 5          & 2          & 4           \\ \hline
\textbf{Team 2} & 5          & 6          & 4          & 1          & 3          & 5          & 6          & 4          & 1          & 3           \\ \hline
\textbf{Team 3} & 4          & 1          & 6          & 5          & 2          & 4          & 1          & 6          & 5          & 2           \\ \hline
\textbf{Team 4} & 3          & 5          & 2          & 6          & 1          & 3          & 5          & 2          & 6          & 1           \\ \hline
\textbf{Team 5} & 2          & 4          & 1          & 3          & 6          & 2          & 4          & 1          & 3          & 6           \\ \hline
\textbf{Team 6} & 1          & 2          & 3          & 4          & 5          & 1          & 2          & 3          & 4          & 5           \\ \hline
\end{tabular}
\caption{Table 2: Timetable for a compact DRR tournament with 6 teams}
\label{tbl2}
\end{table}

\subsection{Schedules}
A HAP set for which a corresponding timetable exists is \textbf{feasible}. The combination of a HAP set and the timetable results in the schedule for the tournament. In table 3 we combine the HAP set of table 1 and the timetable of Table 2 into a schedule for a DRR tournament with 6 teams (Rasmussen 2008). A plus means a home game and minus means an away game. Please note that different representations for the same tournament are possible. \\

\begin{table}[!h]
\centering
\begin{tabular}{|l|l|l|l|l|l|l|l|l|l|l|}
\hline
\textbf{Round}  & \textbf{1} & \textbf{2} & \textbf{3} & \textbf{4} & \textbf{5} & \textbf{6} & \textbf{7} & \textbf{8} & \textbf{9} & \textbf{10} \\ \hline
\textbf{Team 1} & +6         & -3         & +5         & -2         & +4         & -6         & +3         & -5         & +2         & -4          \\ \hline
\textbf{Team 2} & +5         & -6         & -4         & +1         & -3         & -5         & +6         & +4         & -1         & +3          \\ \hline
\textbf{Team 3} & -4         & +1         & +6         & -5         & +2         & +4         & -1         & -6         & +5         & -2          \\ \hline
\textbf{Team 4} & +3         & -5         & +2         & -6         & -1         & -3         & +5         & -2         & +6         & +1          \\ \hline
\textbf{Team 5} & -2         & +4         & -1         & +3         & +6         & +2         & -4         & +1         & -3         & -6          \\ \hline
\textbf{Team 6} & -1         & +2         & -3         & +4         & -5         & +1         & -2         & +3         & -4         & +5          \\ \hline
\end{tabular}
\caption{Timetable for a compact DRR tournament with 6 teams (Rasmussen 2008)}
\label{tbl3}
\end{table}

A schedule for a SRR tournament is irreducible when at least one of the two participants playing a game is on a break. When solving a sport scheduling problem, it could be an advantage to wait assigning different participants to the HAP-set and timetable until the schedule is built. In this case,  placeholders can be used within the HAP set and the timetable [8].
\\[5px]
\textbf{TODO: Discuss NP hard problem here?}


\chapter{Classification and Objective functions}
\section{Classification of a sportscheduling problem}
Solving a sport scheduling problem consists of finding a feasible schedule which is acceptable for the organization of the tournament. This schedule minimizes or maximizes the cost function and does not violate any hard constraints. Common used cost functions are minimizing the amount of breaks or traveling distance. Besides those two, there still are some other things that should be taken into account. For example minimizing costs or maximizing revenue, minimizing carry-over effects  and penalties. We list five complete categories

\subsection{Minimal Break}
Probably the most studied objective in the sport scheduling world is minimizing breaks. Organizers of sport tournaments limit their schedules to having an upper bound for the amount of breaks (e.g x(n-2) for an xRR tournament with an even number of teams n). Pioneering work about breaks has been conducted by Werra(1980,1982). There are several reasons why people want a minimal amount of breaks in their schedule. Fans don’t like long periods without home games from their team(Nurmi et al, 2010), consecutive homegames reduce the revenues from tickets[9]. Long home stands or road trips could also have an impact on the position of a team in the tournament(Nurmi 2010).
\\[5px]
Werra introduced a variance on the minimal break problem in 1982: find a minimal amount of rounds with breaks. An application of a minimal break soft constraint adapted to the demands of the league is the equadorian football league [9 recalde et al]. This function consists of 2 objectives: minimizing the number of breaks and maximizing the number of pseudo-breaks. A pseudo break is a road trip of 2 games where one of the games is played in the own province.
\\[5px] 
It isn’t always the most optimal to minimize the amount of breaks, an increased amount of breaks could reduce the total traveling distance and the travel costs [9], see section 1.2 for more information. If the schedule is mirrored, the minimal amount of breaks will rise to a minimum of 3n-6, breaks are not the main concern for mirrored schedules(Goosens en Spieksma, 2012). Table 3 is a compact mirrored DRR tournament with 6 teams and a minimal number of breaks. No breaks for team 1 and 6, 3 for team 2 and 3, 3 game homestand for team 5 and a 3 game road trip for team 4. Which adds up to 12 breaks (3*6-6). 
\\[5px]
There is also the possibility to keep the amount of breaks the same for all teams to make it more fair. This implicates that the total number of breaks in a SRR would rise to n, and for a DRR tournament to 2n(De Werra 1980). This is called a \textbf{balanced} schedule(Goossens and Spieksma 2012).

\subsection{Minimum Traveling Distance}
The traveling tournament problem(TTP) has been introduced  by Easton et al. (2001). The objective of this problem is to minimize the total distance traveled for each team. This is important when you have a tournament where the home bases have a large distance between them(big travel costs). In a TTP, each team starts in their home base, and return after the last game to their home base. But they don’t always return to their home base after each game, sometimes they continue their road trip from one opponent’s base to another without going back to their home base. Long road trips/home stands are avoided.
\\[5px] 
TTP is stated as followed: Given a number of teams and known distance between different home bases, find a timetable such that the accumulated traveled distance is minimal and other constraints remain satisfied.    
\\[5px]
Urrutia and Ribeiro (2004,2006) proved that minimizing the total traveling distance and maximizing the number of breaks is equivalent if the distance between all home bases equals 1. This is the constant distance traveling tournament problem(CTTP). The motivation for CTTP is that the traveling distance with a maximum number of breaks is a lower bound for the total traveling distance in TTP. The solution for the maximum break problem could be used to find an optimal solution for TTP.

\subsection{Minimal cost  / maximal revenue}
Some literature states that every sport scheduling problem is in this category, because the main concern is to minimize cost and maximize revenue, to generate a certain amount of money while organizing a tournament. But not everyone agrees. Briskorn \& Draxel stated that many approaches  make it possible to generate a schedule with a minimal amount of breaks, but each and everyone of those methods  will fail if the minimization of costs comes into play. They chose to model the minimal cost as the objective function, and to model a maximum amount of breaks as constraints[10]. 
\\[5px]
When the costs are a function we aim fore, we assume that the costs are dependent on the round and on which two participants that are playing. So for each team and in each round, a cost is declared for a home game for each team against an another team. An example of a cost could be the rent of a stadium, and that could be dependent on the season/day of the week, … Revenues could be the economical value of the sold tickets, and those could be dependent on the round and opponent you play.
\\[5px]
Maximizing the supporters rise could also be a objective. This objective could be represented as the economical value of the sold tickets and thus belongs to this category. So we would always try to maximize this function (unless you make the revenues of the tickets negative so it remains a minimizing problem).

\subsection{Minimizing carry over effects}
In a RR schedule, there is a certain sequence where each team meets his opponents. When a team in a round plays against team t1 and in the next round agains team t2, we say that t1 has a carry-over effect onto t2. If t1 is very strong or very weak and several teams play against t1 and t2 right afterwards, then you could say t2 has a (dis)advantage over the other teams. This (dis)advantage could be of importance on psychological level, for example if t1 is very strong, teams that play t1 and lose, could lose trust, giving t2 a psychological advantage. 
\\[5px]
To increase the fairness of the schedule, we try to balance the carry-over effect. In an ideal schedule, this would be that there exist no teams t1,t2,t3,t4, so that t3 and t4 play against t2 immediately after playing t1. If C(t1,t2) is defined as the amount of carry over effects that t1 gives to t2, the COE value (carry-over effect value) of the schedule could be calculated as the sum over all squares of all C(t1,t2). These carry over effects are placed in a carry-over effect matrix. A coe-matrix is a non negative matrix with the following properties: 
\begin{itemize}
\item{Sum of each row is n-1}
\item{Sum of each column is n-1}
\item{The trace is 0}
\end{itemize}
\textbf{**todo include an example from Russel(1980), pending on researchgate**
**TODO2: spieksma states that COE is not present in football, look for additional reseach on this first**}

\subsection{Minimum penalty}
In some cases, no objective gets priority, they aim for a schedule that is fair and that satisfies the given hard constraints. This is the constrained sport scheduling problem (CSSP). We look for a schedule that satisfies all the hard constraints, and minimizes the amount of soft constraint violation. When a soft constraint is violated, a penalty value gets incorporated. The importance of soft constraints are weighted. Those weights are chosen by the organizers of the league and the teams together.
\\[5px] 
Minimum penalty is used in the Brazilian football competition where the revenues with television companies are very important. Ribeiro and Urrutia(2007) came up with a objective function that had two objectives: minimizing the amount of breaks and maximizing the sum of the amount of rounds where at least 1 game of a team from Sao paulo and 1 game from a team from Rio de Janeiro that plays away could be broadcasted. Penalties where incorporated for every break and every round where not a single team from Sao Paulo or Rio were playing away. 
\\[5px]
Some problems maximize a sum of a multiplication of weights and amounts. This is the case when scheduling the football league in Chile where the objective is to maximize the concentration of games between teams from the same group in the last round of the tournament(Duran et al,2007).
\\[5px]
If we make the weights or amounts negative, that maximizing problem becomes a minimization and thus could be placed in the category of minimum penalty.  

\section{Constraints}
To schedule a RR tournament, some constraints are essential. Every RR tournament requires these constraints, they are universal for every RR problem and independent of the problem. Since we have particular interest in scheduling RR tournaments, we will list those constraints here since they have to be satisfied in every RR scheduling problem, but they are ignored in classification.
\\[5px]
\begin{itemize}
\item{Every team plays a game against every other team over a set amount of round}
\item{Every team plays maximum one game every round}
\item{In a SRR, every game exists once, 2RR every game twice, … Two teams play eachother x times in a xRR}
\item{Every team has a home base, and every game is played in a home base of one of the two teams. So one team plays home and the other away (in tournaments on neutral ground, we make use of fictive home and away bases)}
\item{Every team plays in every cycle of the xRR tournament half of his games home and half of the games away (or one more/less if an odd number of teams)}
\item{Two teams play eachother to turn home and away in xRR (with x>1). Two teams can’t play eachother in HH,AA,HHAA,AAHH,HAAH or AHHA. In the list of Nurmi et al. (2010) is this referred to as constraint C22.}

\end{itemize}
Every sport scheduling problem has its own set of (hard) constraints originating from teams, televisioncompanies, sportassociations, fans and local government. Since every league has its own set of constraints, and we want to make a clear distinction between them, we list different constraints that could be applicable on certain sport leagues (and we try to create a general overview).
\\[5px]
In most scheduling problems there is a clear distinction between hard and soft constraints. Hard constraints should be satisfied in a solution, and some solutions should take soft constraints into account. We try to achieve this by minimizing incorporated penalties (see section 1.5). Following list has no distinction between hard and soft constraints, since those are deducted by the stakeholders themselves. 
\\[5px]
The first list of constraints has been introduced by Nurmi(2010) and those constraints were later divided into different categories. The prefix C for the constraints has been changed for other letters who represent the constraint: P for place constraints, GR for group constraints, B for break constraints, GA for game constraints, GE for geographical constraints, Q for tournament quality constraints and S for seperation constraints. A league could use a mix of constraints, and thus not forced to model every constraint. Following parameters are used in this list: team t, round r, amount of round robins x, region g, amounts k \& m , group s, coe value c and day of the week w.


\subsection{Place constraints}
Constraints that make sure that a team t plays home or away in a certain round r.
\\[5px]
\begin{table}[h]
\begin{tabular}{|l|l|}
\hline
P04 & Team t cannot play at home in round r                                                                                              \\ \hline
P05 & Team t cannot play away in round r                                                                                                 \\ \hline
P06 & Team t cannot play at all in round r                                                                                               \\ \hline
P07 & \begin{tabular}[c]{@{}l@{}}There should be at least m1 and at most m2 homegames for teams\\ t1,t2,... on the same day\end{tabular} \\ \hline
P08 & Team t cannot play at home on two consecutive calendar days                                                                        \\ \hline
P23 & Team t wishes to play at least m1 and at most m2 home games 
\\& on weekday1, m3-m4 on weekday2 and so on                               \\ \hline
P39 & The game between Team t1 and team t2 in part x of the xRR
\\& (x odd) tournament has to take place in the home base of team 1          \\ \hline
P40 & Team t wishes to play at least k1 and at most k2 home(or away)
\\ & games between round r1 and round r2                                 \\ \hline
P41 & If a team plays in round r1 home(away), than plays this
\\ & team away in round r2(home)                                                \\ \hline
\end{tabular}
\caption{Place category(Nurmi 2010)}
\label{tbl-3}
\end{table}
These constraints come into play when for example the police has to be present for a big event in the city, and this have an influence on the police preserving security, or the accessibility of a stadium. The as stakeholder could use constraint P04-05-06 to solve that problem. 
\\[5px]
P07 is a constraint that is used if two teams share their home base (pick m1=0 and m2=1), this way they alternate for playing at home. This is why P07 is often used as a hard constraint.
\\[5px]
P08 could be used to force road trips or to balance home advantage,  eg: in a set of 6 consecutive games, t1 has to play 3 home and 3 away games. 
\\[5px]
P23 could be used to schedule some games into the weekends to up the revenue. Or it could be beneficial for teams to play on a certain day. Those preferences could be determined using P23.

\subsection{Group constraints}

It’s common in tournaments that groups are seeded based on strengths or weaknesses or geographic features. Sometimes groups are randomly seeded. Those leagues could have special wishes considering (strength)groups.
\\[5px]
\begin{table}[h]
\begin{tabular}{|l|l|}
\hline
G28 & Teams should not play more than k consecutive games against opponents\\ & in the same strength group                                                                   \\
\hline
G29 & Teams should not play more than k consecutive games against opponents\\ & in the strength group s                                                                      \\
\hline
G30 & At most m teams in strength group s should have a home game in round r                                                                                             \\
\hline
G31 & There should be at least m1 and at most m2 homegames for \\ & teams t1,t2,... on the same day                                                                           \\
\hline
G32 & Team t should play at least m1 and at most m2 home games against \\ & opponents in strength group s 
\\ & between rounds r1 and r2
\\
\hline
\end{tabular}
\caption{Group constraints (Nurmi 2010)}
\label{tbl4}
\end{table}

G28 and 29 could forbid teams to play 5 consecutive games against weak or elite teams (teams from big cities who have a lot of revenues from television). 
\\[5px]
A television broadcaster could request that not all the big teams play a home game at the same moment using G30. 
\\[5px]
G31 could be used right before the qualification for playoffs, this way you could make sure that it will take until the last round before playoffs until all qualifying teams are known. 
\\[5px]
G32 could be used to ensure that each team plays in the first x round a game agains a top ranked team. 

\subsection{Break constraints}
Minimizing the amount of breaks is ususally an objective when scheduling round robin. The amount of breaks in the schedule is very important for the fairness of the tournament.


\begin{table}[h]
\begin{tabular}{|l|l|}
\hline
B12 & A break cannot occur in round r                               \\
\hline
B13 & Teams cannot have more than k consecutive home games          \\
\hline
B15 & The total number of breaks must not be larger than k          \\
\hline
B16 & The total number of breaks per team must not be larger than k \\
\hline
B17 & Every team must have an even number of breaks   				  \\             
\hline
\end{tabular}
\caption{Break constraints (Nurmi 2010)}
\label{tbl5}
\end{table}

B12 could ensure that there are no breaks in the last round
\\[5px]
B13 is used to increment the fairness of the tournament, this constraint is often incorporated as a soft constraint: no team can play two consecutive home/away games. If we incorporate k=1, this constraint states that the amount of breaks should be minimal. 





\chapter{Representation of Sport Tournaments}
\input{chapters/4.RepresentationofSportTournaments.tex}

\chapter{Related Work}
\input{chapters/5.RelatedWork.tex}
\end{document}

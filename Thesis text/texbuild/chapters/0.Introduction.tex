Many real world problems could be easily solved using constraints. In this paper we take a particular interest in the world of sport scheduling. The sport scheduling problem consists of planning games in a sport tournament that satisfy a certain amount of constraints stated by different stakeholders within the tournament (organizers, participants, television rights, fans, government, …).
\\[5px]
A wide variety of scheduling models exists with regards to formats of sporting tournaments. A brief summary will be given, but the main focus of this paper will be on round robin tournaments due to their simplicity and resulting from that, the fact that they are very commonly used in different sports. A round robin tournament is a tournament where each player or team plays each one of the other players or teams a fixed number of times.
\\[5px]
Apart from planning when teams play each other, the location is also important. A core aspect of sport scheduling is the creation of schedules which are optimized for logistics and offer every stakeholder the same amount of (dis)advantages.
\\[5px]
Trying to obtain a realistic model that is both optimized as well as satisfying all the constraints of the stakeholders is very hard and labour-intensive [1]. An alternative to solving this problem could be the employment of constraint learning to induce a model based on given positive examples. A given model could be used “as is” to produce solutions in line with the problem, or serve as a base model to design the final model.
\\[5px]
There are many artificial intelligence approaches that could solve the components of this problem. The hard constraints stated by the stakeholders could easily be solved with basic constraint programming, while the optimizations could be satisfied using integer programming [2]. The sport scheduling problem involves numerical constraints. For instance, the number of consecutive home or away games for a certain team could be subject to a constraint.  Classical learning approaches like Conacq and Inductive Logic Programming focus on boolean variable, and it -s unclear how we could introduce numerical terms in here.
\\[5px]
The sport constraint problem is inherently multi-dimensional, so the constraint learning approach used to solve the sport scheduling problem is based on COUNT-OR [3]. This approach is used as a benchmark. The adaptation Count-SPORT is designed as a generalization of Count-OR, trying to acquire hard constraints like “team 2 plays not more than 2 consecutive games at home”, but also tries to optimize schedules based on soft constraints.
